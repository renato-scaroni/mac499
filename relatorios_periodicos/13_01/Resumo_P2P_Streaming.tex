\IfFileExists{preamble.tex}
{\input{preamble.tex}}
{
% \documentclass[a4paper,12pt]{article}
\documentclass[a4paper]{report} %padrao letterpaper, 10pt

\usepackage[brazil]{babel}
%\ usepackage[utf8]{inputenc}
\usepackage[latin1]{inputenc}
\usepackage[T1]{fontenc}
\usepackage{url}
\usepackage{tabularx}
\usepackage{array} % for defining a new column type
\usepackage{varwidth} %for the varwidth minipage environment
\usepackage{amsfonts,amssymb,graphicx,enumerate}
\usepackage[centertags]{amsmath}
\usepackage{hyperref}
\usepackage{setspace}
\usepackage{fancyhdr}
\usepackage{indentfirst}
\usepackage{alltt}
\usepackage{amssymb}
\usepackage{amsmath}
\usepackage{amstext}
\usepackage{amsfonts}
\usepackage{color}
\usepackage{listings}
\usepackage{tipa}
% Configuracoes de pagina
\usepackage[lmargin=3cm,rmargin=3cm,tmargin=3cm,bmargin=3cm]{geometry}
% Layout da pagina
\usepackage{hyperref}
%\usepackage{qtree}
\usepackage{vaucanson-g}
\usepackage{enumitem}
\usepackage{fixltx2e}
}


\begin{document}
\newcolumntype{A}{>{\begin{varwidth}{2cm}}l<{\end{varwidth}}} %M is for Maximal column
\newcolumntype{B}{>{\begin{varwidth}{4cm}}l<{\end{varwidth}}} %M is for Maximal column
\newcolumntype{C}{>{\begin{varwidth}{2cm}}l<{\end{varwidth}}} %M is for Maximal column
\newcolumntype{D}{>{\begin{varwidth}{2cm}}l<{\end{varwidth}}} %M is for Maximal column
\newcolumntype{E}{>{\begin{varwidth}{2cm}}l<{\end{varwidth}}} %M is for Maximal column

\begin{center}
  \Large\textbf{Resumo sobre P2P e v�deo live streaming}
\end{center}



\bigskip


\begin{flushright}
  \emph{Jan/2014}
\end{flushright}

\bigskip


% \section{Exerc�cios Pr�ticos}

% \begin{table}[h]


\begin{enumerate} %come�o da lista
  \item \textbf{Video live streaming atualmente} \\ \\ 
   
   Atualmente existem poucas alternativas de streaming de v�deo realtime para uma grande audi�ncia
   acess�veis ao p�blico geral. As principais alternativas existentes s�o o Upstream e o Justin TV,
   e em ambos os casos o mecanismo usado para fazer o broadcast consiste basicamente em o usu�rio
   enviar seus dados para os servidores do servi�o. L� o stream do usu�rio ser� repassado para outros 
   servidores que retransmitir�o o v�deo para os usu�rios que desejem assist�-lo. A quantidade de
   servidores destinados a fazer o broadcast depende da quantidade de usu�rios conectados e a aloca��o
   de quais servidores retransmitir�o cada stream de dados depende da implementa��o. No caso do
   Justin TV, existe um sistema que monitora o stream de entrada e sa�da para cada broadcast e realoca 
   os servidores de acordo com a quantidade e localiza��o de espectadores para um dado canal (o Justin TV
   possui v�rios datacenters posicionados nas regi�es mais populosas dos Estados Unidos de forma a atender
   ao maior n�mero de espectadores poss�vel com a maior efici�ncia poss�vel). A transmiss�o de v�deo ocorre 
   atrav�s do protocolo RTMP, baseado em flash e que utiliza conex�es TCP para a transmiss�o.
      

\end{enumerate}
% \end{table}

\begin{thebibliography}{99}
\bibitem{Justin.tv} T. Hoff,  "Justin.tv's Live Video Broadcasting Architecture", http://highscalability.com/blog/2010/3/16/justintvs-live-video-broadcasting-architecture.html
\bibitem{BitLIve} B. Cohen, BitTorrent Live official patent document, http://www.scribd.com/doc/132418122/bittorrent-live-patent
\bibitem{RTMP} "Real-Time Messaging Protocol (RTMP) specification", http://www.adobe.com/devnet/rtmp.html
\end{thebibliography}

\addcontentsline{toc}{chapter}{Refer\^encias Bibliogr\'aficas}

\end{document}